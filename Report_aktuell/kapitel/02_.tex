\chapter{Lösungsansatz} 
Das Paper stellt einen algorithmischen Ansatz vor, der das Problem für eine gegebene Konfiguration löst. Der Algorithmus arbeitet in sogenannten \emph{'Runden'}, wobei jede Runde aus mehreren Schritten besteht. Der Ausgangspunkt jeder Runde ist die Sortierung der Wassermengen in den Krügen, gefolgt von der Berechnung bestimmter Parameter, die die nächsten Schritte bestimmen. \\
Es werden zwei Hauptfälle unterschieden, die je nach Verhältnis der Wassermengen in den Krügen unterschiedlich behandelt werden. \\ 
Der Algorithmus ist so konstruiert, dass er im schlimmsten Fall eine obere Schranke von $O(log n)$ für die Anzahl der erforderlichen Schritte erreicht, wobei n die Gesamtliterzahl des Wassers in den Krügen ist.  %ToDo stimmt das?

\section{Der Ablauf innerhalb einer Runde} \label{runden} 
Eine Runde startet mit einer Anfangskonfiguration $(a, b, c)$ mit $0 < a\leq b \le c$ und endet mit einer Konfiguration $(a', b', c')$ mit $a' \leq a/2$. Wenn eine der drei Zahlen nach einer Runde 0 ist, ist das Ziel erreicht. \\
Eine Runde ist immer folgendermaßen aufgebaut:

\begin{itemize}
    \item Sortieren der Krüge, sodass $a \leq b \leq c$
    \item die Hilfswerte $r$, $p$, $q$ berechnen
    \item Fall 1 oder Fall 2 ausführen je nachdem ob $b-pa \leq a/2$ gilt.
    \item Prüfen ob wir unser Ziel erreicht haben
\end{itemize}  

\section{Berechnung der Hilfswerte r, p und q} \label{helper}
Die Hilfswerte werden folgendermaßen berechnet:
    $r \coloneq b/a$, 
    $p \coloneq \rfloor r \rfloor $ und
    $q \coloneq \lceil r \rceil $.
Dabei können wir etwas festellen. Es gilt: $0 \leq b-pa < a $ und $ 0 \leq qa - b < a $. 
Das impliziert dass $min(b-pa, qa-b) \leq a/2$, da $(b-pa)+(qa-b)=(q-p)a \leq a$. 
Außerdem betrachten wir die kleinsten binären Repräsentationen von $p$ und $q$: $p_k...p_0$ und $q_\ell...q_0$. Diese werden im Ablauf von Fall 1 und Fall 2 relevant.
\section{Fall 1} \label{case-1}
Aus den Hilfswerten prüfen wir die Bedingung $b-pa \leq a/2$. Wenn sie eintritt, dann führen wie Fall 1 aus, wenn nicht, dann Fall 2. 
In Fall 1 führen wir $k+1$ Schritte für $i=0, ..., k$ aus, die immer das verdoppeln werden was anfänglich $a$ war. Dazu muss erst $a$, dann $2a$ und allgemein $2^ia$ von einer der anderen Nummern abgezogen werden. 
Wir ziehen den benötigten Betrag 
\begin{itemize}
    \item von der zweiten Nummer ab (anfänglich $b$), wenn $p_i = 1$ und
    \item von der dritten Nummer ab (anfänglich $c$), wenn $p_i = 0$.
\end{itemize}

Nach der Ausführung dieser $k+1$ Schritte, sind die zweite und die dritte Nummer $b- \sum_{i=0}^{k} p_i2^ia = b-pa \leq a/2$ bzw. $c- \sum_{i=0}^{k} (1-p_i)2^ia$.

\section{Fall 2} \label{case-2}
% TODO: die ander Bedingung die dann auch erfüllt ist???
Wird die Bedingung $b-pa \leq a/2$ nicht erfüllt, dann führen wie Fall 2 aus. 
In Fall 2 führen wir $\ell$ Schritte für $i=0, ..., \ell-1$ aus, die immer das verdoppeln werden was anfänglich $a$ war. Dazu muss wie in Fall 1 erst $a$, dann $2a$ und allgemein $2^ia$ von einer der anderen Nummern abgezogen werden. 
Anders als in Fall 1 jedoch ziehen wir den benötigten Betrag 
\begin{itemize}
    \item von der zweiten Nummer ab (anfänglich $b$), wenn $q_i = 1$ und
    \item von der dritten Nummer ab (anfänglich $c$), wenn $q_i = 0$.
\end{itemize}

Nach der Ausführung dieser $\ell$ Schritte, ist die erste Nummer $2^\ell a$, die zweite $b- \sum_{i=0}^{\ell-1} q_i2^ia = b-(q-2^\ell)a$ und die dritte Nummer $c- \sum_{i=0}^{\ell-1} (1-q^i)2^ia$. \\

Anschließend führen wir noch einen $(\ell +1)$-ten Schritt aus. Dieser macht aus $a = 2^\ell a- (b-qa+2^\ell a) = (-b+qa)$. \\

Die Runde endet und die erste Numer ist nun genau $qa-b < q/2$.